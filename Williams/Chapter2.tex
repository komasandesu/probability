\documentclass{jsarticle}
\usepackage[utf8]{inputenc}

\usepackage[dvipdfmx]{graphicx}
\usepackage{bm}
\usepackage{amsmath,amsthm,amsfonts,amssymb}
\usepackage{enumitem}
\usepackage{mathrsfs}
\usepackage{mathtools}
\usepackage{latexsym}
\usepackage{ascmac}

\renewcommand{\proofname}{\textbf{\rm 証明}}


\title{Williams Chapter 2}
\author{koma}
\date{\today}

\begin{document}
\maketitle

\subsubsection*{2.1}
実験用モデル$(\Omega,\mathcal{F},\mathbf{P})$

ランダム性を伴う実験のモデルは,セクション1.5の確率空間$(\Omega,\mathcal{F},\mathbf{P})$の形式を取る.

\begin{description}
    \item[\bf 標本空間] 
        $\Omega$は標本空間と呼ばれる.
    \item[\bf 標本点] 
        $\Omega$の$\omega$は標本点と呼ばれる.
    \item[\bf 事象] 
        $\Omega$上の$\sigma$加法族$\mathcal{F}$は事象族と呼ばれ,$\Omega$の$\mathcal{F}$可測集合である$\mathcal{F}$の要素は事象となります.
\end{description}

確率空間の定義により$\mathbf{P}$は$(\Omega,\mathcal{F})$の確率測度です。

\subsubsection*{2.2}
直感的な意味

チャンスの女神テュケは,法則$\mathbf{P}$に従って$\Omega$の点$\omega$を「ランダムに」選択する.
つまり,$\mathcal{F}$の$F$について,$\mathbf{P}(F)$は(私たちの直観によって理解される意味で) 「確率」を表す.
テュケが選んだ点$\omega$は$F$に属する.

選択した点$\omega$によって実験の結果が決まるので,以下のような写像が存在する.
\begin{align}
    &\Omega \rightarrow 結果の集合 , \nonumber \\
    &\omega \rightarrow 結果\nonumber
\end{align}

この「写像」は必ずしも1対1対応である必要がある理由はない.実験には明らかな「最小」または「標準」モデルがあるにもかかわらず、より豊富なモデルを使用する方が良い場合がよくある.

たとえば,ランダムウォークをブラウン運動に埋め込むことで,コイン投げの多くの特性を読み取ることができる.

\subsubsection*{2.3}
$(\Omega,\mathcal{F})$の例
\begin{description}
    \item[(a)] 実験:コインを二回投げる.すると
        \begin{equation}
            \Omega=\{HH,HT,TH,TT\} , \mathcal{F}=\mathcal{P}(\Omega)\coloneqq \Omega の全てのすべての部分集合の集合族. \nonumber
        \end{equation}
        このモデルでは,「少なくとも1回の結果が表になる」という直感的な事象は,数学的な事象 ($\mathcal{F}$ の要素) $\{HH,HT,TH\}$ によって記述される.
    \item[(b)] 実験:コインを無限に投げる.すると
        \begin{equation}
            \Omega=\{H,T\}^{\mathbf{N}} \nonumber
        \end{equation}
        であるので$\Omega$の$\omega$は以下のように書くことができる.
        \begin{equation}
            \omega = (\omega_1,\omega_2,\dots) , \omega_n \in \{H,T\}. \nonumber
        \end{equation}

        ここで,直感的な事象$\omega_n=W’,W\in\{H,T\}$について
        \begin{equation}
            \mathcal{F} = \sigma(\{\omega\in\Omega \colon n\in\mathbf{N},W\in\{H,T\}) \nonumber
        \end{equation}
        のように自然に決めることができる.

        $\mathcal{F}\neq\mathcal{P}(\Omega)$だが,$\mathcal{F}$は十分に大きいことがわかる.
        例えば,3.7節では
        \begin{equation}
            \frac{ n回のトスでの表の数 }{n} \to \frac{1}{2}
        \end{equation}
        という文の真理集合
        \begin{equation}
            F=\left\{ \omega\colon \frac{\#(k\leq n\colon\omega_k=H)}{n} \to \frac{1}{2} \right\} \nonumber
        \end{equation}
        が$\mathcal{F}$の要素となることがわかる.

        なお,(a)の実験では,サンプル点と結果の写像$\omega\mapsto(\omega_1,\omega_2)$を用いて,モデルをより情報量の多いモデルとして使うことができる.

        \item[(c)]実験:0と1の間の点を一様に無作為に選ぶ.
        選択された点を表す$\Omega=[0,1],\mathcal{F}=\mathcal{B}[0,1],\omega$を考える.この場合,明らかに$\mathbf{P}={\rm Leb}$となる.
        このモデルが公正なコインのモデル(b)を含む意味の説明については,後とする.
\end{description}

\subsubsection*{2.4}
Almost surely(a.s.)

結果に関する状態$S$は次の場合,ほとんど確実に(a.s.),または確率1(w.p.1)で真であると言われる.
\begin{equation}
    F \coloneqq \{\omega \colon S(\omega)は真\}\in\mathcal{F} かつ \mathbf{P}(F)=1 .
\end{equation}

(a)命題 $F_n\in\mathcal{F}(n\in\mathbf{N})$かつ$\forall n,\mathbf{P}(F_n)=1$であれば$\mathbf{P}(\bigcap_n F_n)=1$である.

\begin{proof}
    $\forall n,\mathbf{P}(F_n^c)=0$であるので1.10(c)より
    $\mathbf{P}(\bigcup_n F_n^c)=0$となる.

    ここで,$\bigcap_n F_n=(\bigcup_n F_n^c)^c$より
    $\mathbf{P}(\bigcap_n F_n)=1$である.
\end{proof}

\begin{itembox}{$\rm \mu \text{-} null\ set$との違い}
    \begin{description}
        \item[$\rm almost\ surely$ :] 命題が真であること ((a):可算個のa.s.$\Rightarrow$ a.s.)
        \item[$\rm \mu \text{-} null\ set$ :] 命題が偽であること (真であることについて考えると測度が$\infty$になる場合があるため)
    \end{description}
\end{itembox}

\subsubsection*{2.5}
注意 :  $\limsup, \liminf, \downarrow\lim$等

\begin{description}
    \item[(a)] $(x_n : n\in\mathbf{N})$を実数列とする.
        \begin{equation}
            \limsup x_n \coloneqq \inf_m\left\{ \sup_{n\geq m}x_n \right\} = \downarrow\lim_m\left\{ \sup_{n\geq m}x_n \right\} \in [-\infty,\infty]
        \end{equation}
        のように定義する.
        $y_m\coloneqq \sup_{n\geq m}x_n$は広義単調減少であるので,$y_m$の極限は$[-\infty,\infty]$に必ず存在する.

        \begin{itembox}{}
            $y_m \coloneqq \sup\{x_n:n\geq m\}$であり,
            $\{x_n:n\geq m\} \supset \{x_n:n\geq m+1\}$より広義単調減少
        \end{itembox}
        
        単調な極限を意味する$\uparrow\lim$や$\downarrow\lim$は便利で,$y_{\infty}=\downarrow\lim y_n$を意味する$y_n \downarrow\ y_{\infty}$の使用も便利である.
    \item[(b)] 同様にして
        \begin{equation}
            \liminf x_n \coloneqq \sup_m\left\{ \inf_{n\geq m}x_n \right\} = \uparrow\lim_m\left\{ \inf_{n\geq m}x_n \right\} \in [-\infty,\infty]
        \end{equation}
    \item[(c)] 
        $x_n$が$[-\infty,\infty]$に収束する $\Longleftrightarrow$ $\limsup x_n=\liminf x_n$
        このとき,$\lim x_n=\limsup x_n=\liminf x_n$
    \item[(d)] \mbox{}
        \begin{description}
            \item[(i)] $z > \limsup x_n$の場合,
                最終的に(つまり,十分に大きいすべての$n$について)$x_n<z$となる.
                \begin{itembox}{}
                    \begin{equation}
                        z>\limsup x_n=\lim_{m\to\infty}\sup_{n\geq m}x_n \Longleftrightarrow \exists m\in\mathbf{N} s.t.\forall n\in\mathbf{N},n\geq m, z>x_n \nonumber
                    \end{equation}
                \end{itembox}
            \item[(ii)] $z < \limsup x_n$の場合,
                無限回(つまり,無限に多くの$n$について)$x_n>z$となる.
                \begin{itembox}{}
                    \begin{equation}
                        z<\limsup x_n=\lim_{m\to\infty}\sup_{n\geq m}x_n \Longleftrightarrow \forall m\in\mathbf{N} s.t.\exists n\in\mathbf{N},n\geq m, x_n>z \nonumber
                    \end{equation}
                \end{itembox}
        \end{description}
\end{description}

\subsubsection*{2.6}
定義 $\limsup E_n,(E_n,{\rm i.o.})$

事象(真理集合)「表が出た回数/投げた回数 $\to \frac{1}{2}$」は,「$n$回目の試行で表が出る」といった単純な事象から、かなり複雑な方法で構築されます。」

集合の$\liminf$と$\limsup$を取るという考え方により示す.

$E$がである場合、次のトートロジーに注意すると役立つかもしれない.
\begin{equation}
    E=\{ \omega : \omega \in E \} \nonumber
\end{equation}

ここで$(E_n \colon n\in N)$が事象列であると仮定する.

\begin{description}
    \item[(a)]以下のように定義する.
        \begin{align}
            (E_n,{\rm i.o.}) &\coloneqq (E_nが無限回起こる) \nonumber \\
            &\coloneqq \limsup E_n \coloneqq \bigcap_m\bigcup_{n\geq m}E_n \nonumber \\
            &=\{\omega \colon 任意の m\in\mathbf{N} について, \exists n(\omega)\geq m( ただし, \omega\in E_{n(\omega)} である ) \} \nonumber \\
            &=\{\omega \colon 無限に多くの n について \omega \in E_n \} \nonumber
        \end{align}
        \begin{itembox}{}
            $\limsup E_n \coloneqq \bigcap_{m\in\mathbf{N}}\bigcup_{n\geq m}E_n$ の意味
            \begin{align}
                &\omega \in \bigcap_{m\in\mathbf{N}}\bigcup_{n\geq m}E_n \nonumber \\
                \Leftrightarrow &\forall m\in\mathbf{N} , \omega\in \bigcup_{n\geq m}E_n \nonumber \\
                \Leftrightarrow &\forall m\in\mathbf{N} , \exists n\geq m , \omega\in E_n \nonumber
            \end{align}
            つまり,$m\in\mathbf{N}$がどんなに大きくても$\exists n\in\mathbf{} {\rm s.t.}n\geq m,\omega\in E_n$
        \end{itembox}
    \item[(b)] 逆Fatouの補題($\mathbf{P}$の有限性が必要)
        \begin{equation}
            \mathbf{P}(\limsup E_n) \geq \limsup\mathbf{P}(E_n) . \nonumber
        \end{equation}
        \begin{proof}
            $G_m\coloneqq\bigcup_{n\geq m}E_n$とする.
            このとき,$G_m\downarrow G (G\coloneqq\limsup E_n)$である.よって1.10(b)より,$\mathbf{P}(G_m)\downarrow\mathbf{P}(G)$となる.
            
            ここで,明らかに
            \begin{equation}
                \mathbf{P}(G_m) \geq \sup_{n\geq m}\mathbf{P}(E_n) . \nonumber
            \end{equation}
            したがって,
            \begin{equation}
                \mathbf{P}(G) \geq \downarrow\lim_m\left\{ \sup_{n\geq m}\mathbf{P}(E_n) \right\} \eqqcolon \limsup\mathbf{P}(E_n) . \nonumber 
            \end{equation}

            \begin{itembox}{}
                \begin{equation}
                    \forall\varepsilon>0 , \mathbf{P}(G_m)>\sup_{n\geq m}\mathbf{P}(E_n)-\varepsilon \nonumber
                \end{equation}
                が成り立つので,$\varepsilon\downarrow 0$を考えると
                \begin{equation}
                    \mathbf{P}(G_m)>\sup_{n\geq m}\mathbf{P}(E_n) \nonumber
                \end{equation}
                が成立する.
            \end{itembox}
        \end{proof}
\end{description}

\subsubsection*{2.7}
Borel–Cantelliの補題

$(E_n \colon n\in\mathbf{N})$を$\sum_n\mathbf{P}(E_n)<\infty$であるような事象列とする.
このとき,以下が成り立つ.
\begin{equation}
    \mathbf{P}(\limsup E_n) = \mathbf{P}(E_n,{\rm i.o.}) = 0 . \nonumber
\end{equation}

\begin{proof}
    2.6(b)のように$G_m,G$を表すとする.
    1.9(b),1.10(a)より,各$m\in\mathbf{N}$において
    \begin{equation}
        \mathbf{P}(G) \leq \mathbf{P}(G_m) \leq \sum_{n\geq m}\mathbf{P}(E_n) \nonumber
    \end{equation}
    が成り立つ.
    ここで$m\uparrow\infty$とすれば示すことができる.
    \begin{itembox}{}
        \begin{equation}
            z>\limsup x_n=\lim_{m\to\infty}\sup_{n\geq m}x_n \Longleftrightarrow \exists m\in\mathbf{N} s.t.\forall n\in\mathbf{N},n\geq m, z>x_n \nonumber
        \end{equation}
    \end{itembox}
\end{proof}

\subsubsection*{2.8}
定義 $\liminf E_n,(E_n,{\rm ev})$

もう一度$(E_n \colon n\in N)$が事象列であるとする.
\begin{description}
    \item[(a)]以下のように定義する.
        \begin{align}
            (E_n,{\rm ev}) &\coloneqq (E_nが最終的に起こる) \nonumber \\
            &\coloneqq \liminf E_n \coloneqq \bigcup_m\bigcap_{n\geq m}E_n \nonumber \\
            &=\{\omega \colon ある m\in\mathbf{N} について, \forall n\geq m(\omega) , \omega\in E_{n} が成り立つ \} \nonumber \\
            &=\{\omega \colon すべての大きな n に対して \omega \in E_n \} \nonumber
        \end{align}
    \item[(b)] $(E_n,{\rm ev})^c = (E_n^c,{\rm i.o.})$である.
    \item[(c)] Fatouの補題(任意の測度空間で成り立つ)
        \begin{equation}
            \mathbf{P}(\liminf E_n) \leq \liminf\mathbf{P}(E_n) . \nonumber
        \end{equation}
        \begin{proof}
            $G_m\coloneqq\bigcap_{n\geq m}E_n$とする.
            このとき,$G_m\uparrow G (G\coloneqq\liminf E_n)$である.よって1.10(b)より,$\mathbf{P}(G_m)\uparrow\mathbf{P}(G)$となる.
            
            ここで,明らかに
            \begin{equation}
                \mathbf{P}(G_m) \leq \inf_{n\geq m}\mathbf{P}(E_n) . \nonumber
            \end{equation}
            したがって,
            \begin{equation}
                \mathbf{P}(G) \leq \uparrow\lim_m\left\{ \inf_{n\geq m}\mathbf{P}(E_n) \right\} \eqqcolon \liminf\mathbf{P}(E_n) . \nonumber 
            \end{equation}
        \end{proof}
\end{description}



\end{document}